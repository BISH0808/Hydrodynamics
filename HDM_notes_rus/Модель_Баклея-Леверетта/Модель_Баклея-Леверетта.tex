\documentclass[a4paper,14pt]{extarticle}
\renewcommand{\baselinestretch}{1.2}
\usepackage[utf8]{inputenc}
\usepackage[T2A, T1]{fontenc}
\usepackage[english, russian]{babel}

\usepackage{fontspec}
\setmainfont{Times New Roman}
\usepackage{setspace, amsmath}
\usepackage{amssymb}
\usepackage{dsfont}
\usepackage{epsfig}
\usepackage{pdfpages}

\makeatletter
\let\@fnsymbol\@arabic
\makeatother

\usepackage{geometry}
\geometry{
a4paper,
total={180mm, 257mm},
left=15mm,
top=10mm,
}

\usepackage{systeme}
\usepackage{skak}
\usepackage{mathtools}
\usepackage{unicode-math}
\usepackage{array}
\usepackage{makecell}
\usepackage{subfiles}
\usepackage{hyperref}
\hypersetup{pdfstartview=FitH, linkcolor=black, urlcolor=blue, colorlinks=true}
\usepackage{framed}
\usepackage{graphicx}
\usepackage{caption}
\usepackage{subcaption}
\usepackage{color}
\usepackage{chngcntr}
\usepackage{tikz}
\usepackage{csquotes}
\usepackage{fancyhdr}
\usepackage{fancyvrb}
\usepackage{comment}
\usepackage{adjustbox}
\usepackage[breakable, skins]{tcolorbox}


\usepackage{float}
\floatstyle{plaintop}
\usepackage{enumitem}
\setlength{\parindent}{0pt}
%\setcounter{section}{-1}

\graphicspath{{./img/}}
\newcommand{\myPictWidth}{.95\textwidth}
\newcommand{\phm}{\phantom{-}}
\newcommand{\beq}{\begin{equation}}
\newcommand{\eeq}{\end{equation}}


\begin{document}

\section*{Модель Баклея-Леверетта}

Исторически первой появилась не модель Баклея-Леверетта, а модель поршневого вытеснения.

Модель поршневого вытеснения далеко не совершенна и может быть использована для очень приближённых (оценочных) расчётов (например, для верхней оценки времени прихода фронта заводнения -- фронт достигнет определённого положения не дольше, чем за время поршневого вытеснения).
\\

\textbf{Основные допущения модели поршневого вытеснения:}

1) рассматривается несмешивающаяся фильтрация (т.е. существует явный раздел фаз);

2) жидкости и порода несжимаемы;

3) за фронтом вытеснения остаётся только остаточная нефтенасыщенность
\\

При поршневом вытеснении граница фронта заводнения фактически определяется из накопленной закачки воды.
\\

Достаточно быстро заметили, что с приходом фронта воды обводнились не сразу до 100\%, а продолжаем добывать нефть.
Это говорит о том, что несмотря на прорыв фронта за этим фронтом осталось достаточно много запасов (не только остаточная нефтенасыщенность).
И эти запасы постепенно вымываются. Чтобы учесть подобное поведение, была разработана \textbf{модель Баклея-Леверетта}.
\\

\textbf{Основные допущения модели Баклея-Леверетта:}

1) рассматривается несмешивающаяся фильтрация (т.е. существует явный раздел фаз);

2) жидкости и порода несжимаемы (нет зависимости плотностей флюидов и пористости от давления и времени);

3) жидкости движутся в соответствии с фазовыми проницаемостями (т.е. для каждой жидкости выполняется закон Дарси, но у каждой из жидкостей своя эффективная проницаемость);

4) капиллярное давление равно нулю (т.е. $p_o=p_w=p$)

С учётом введённых допущений рассмотрим двухфазную фильтрацию системы вода-нефть в пористой среде.

\textbf{Закон Дарси для нефти} (ось $x$ направлена вдоль направления фильтрации; силами гравитации пренебрегаем):
\beq\label{DarcyOil}
\varphi\cdot v_o=-\frac{k\cdot k_{ro}(S_w)}{\mu_o}\cdot\frac{\partial p}{\partial x}
\eeq

\textbf{Закон Дарси для воды} (ось $x$ направлена вдоль направления фильтрации; силами гравитации пренебрегаем):
\beq\label{DarcyWater}
\varphi\cdot v_w=-\frac{k\cdot k_{rw}(S_w)}{\mu_w}\cdot\frac{\partial p}{\partial x}
\eeq

Из \eqref{DarcyOil} и \eqref{DarcyWater}:
\beq\label{Buckley_Leverett_func}
\dfrac{v_w}{v_o+v_w}=\dfrac{\dfrac{k_{rw}(S_w)}{\mu_w}}{\dfrac{k_{ro}(S_w)}{\mu_o}+\dfrac{k_{rw}(S_w)}{\mu_w}}=\frac{1}{1+\dfrac{k_{ro}\mu_w}{k_{rw}\mu_o}}\stackrel{\text{def}}{=}f_w(S_w)
\eeq

получаем определение \textbf{функции Баклея-Леверетта} $f_w(S_w)$ (зависит от насыщенности, так как относительные фазовые проницаемости, входящие в определение, зависят от насыщенности).

\textbf{Закон сохранения массы (неразрывности потока) для нефти} (используем допущение о несжимаемости нефти; ось $x$ направлена вдоль направления фильтрации):
\beq\label{ConstMassOil}
\varphi\rho_{o}\frac{\partial\left(1-S_w\right)}{\partial t}+\rho_o\frac{\partial v_o}{\partial x}=0
\eeq

\textbf{Закон сохранения массы (неразрывности потока) для воды} (используем допущение о несжимаемости воды; ось $x$ направлена вдоль направления фильтрации):
\beq\label{ConstMassWater}
\varphi\rho_{w}\frac{\partial S_w}{\partial t}+\rho_w\frac{\partial v_w}{\partial x}=0
\eeq

Складывая \eqref{ConstMassOil} и \eqref{ConstMassWater}, получаем:
\beq\label{ZeroSumDiv}
\frac{\partial\left(v_o+v_w\right)}{\partial x}=0
\eeq
(т.е. суммарная скорость двухфазного потока не зависит от координаты $x$: либо является постоянной величиной, либо известной функцией времени).

Выражая $v_w$ из \eqref{Buckley_Leverett_func} и подставляя её в \eqref{ConstMassWater}, получаем:
\beq
\varphi\frac{\partial S_w}{\partial t}+\frac{\partial\!\left(f_w(S_w)\cdot(v_o+v_w)\right)}{\partial x}=0
\eeq

Раскроем производную произведения:
\beq
\varphi\frac{\partial S_w}{\partial t}+\frac{\partial\!\left(f_w(S_w)\right)}{\partial x}\cdot\left(v_o+v_w\right)+f_w(S_w)\cdot\frac{\partial\left(v_o+v_w\right)}{\partial x}=0
\eeq

Учитывая равенство \eqref{ZeroSumDiv}, получаем:
\beq\label{Buckley_Leverett_eqn}
\varphi\frac{\partial S_w}{\partial t}+\frac{\partial\!\left(f_w(S_w)\right)}{\partial x}\cdot\left(v_o+v_w\right)=0
\eeq

По правилу цепочки дифференцирования сложной функции:
$$
\frac{\partial\!\left(f_w(S_w)\right)}{\partial x}=\frac{\partial f_w}{\partial S_w}\frac{\partial S_w}{\partial x}
$$

Суммарная скорость двухфазного потока $\left(v_o+v_w\right)$ есть отношение расхода жидкости $q_t$ к площади поперечного сечения фильтрации $A$.

Таким образом, из \eqref{Buckley_Leverett_eqn} получаем основное уравнение двухфазной фильтрации (уравнение Баклея-Леверетта) в классическом виде:
\beq\label{Buckley_Leverett_eqnClassic}
\varphi\frac{\partial S_w}{\partial t}=-\frac{\partial f_w}{\partial S_w}\frac{\partial S_w}{\partial x}\frac{q_t}{A}
\eeq

В процессе нагнетания воды в пласт её насыщенность будет меняться со временем вдоль направления движения $x$.
Связь между $S_w$, $x$ и $t$ можно записать в функциональной форме $S_w=S_w(x,t)$ или в дифференциальной форме:
\beq
dS_w=\frac{\partial S_w}{\partial x}dx+\frac{\partial S_w}{\partial t}dt
\eeq

Рассмотрим на плоскости $(x,t)$ такие линии $x(t)$, вдоль которых насыщенность принимает заданное постоянное значение (т.е. рассмотрим изолинии насыщенности).
Для любого заданного значения насыщенности $S_{w0}$ можно установить такую связь между $x$ и $t$, что выполняется равенство $S_w=S_w(x,t)=S_{w0}=\text{const}$ или в дифференциальной форме:
\beq\label{ConstSaturationDiff}
\frac{\partial S_w}{\partial x}dx+\frac{\partial S_w}{\partial t}dt=0
\eeq

Таким образом, линия (фронт) распространения с заданной водонасыщенностью (т.е. связь между $x$ и $t$ при фиксированном значении $S_{w0}$) получается совместным решением уравнений \eqref{Buckley_Leverett_eqnClassic} и \eqref{ConstSaturationDiff}:
\beq\label{Front}
\begin{cases}
\varphi\dfrac{\partial S_w}{\partial t}+\dfrac{\partial f_w}{\partial S_w}\dfrac{\partial S_w}{\partial x}\dfrac{q_t}{A}=0 \\\\
\dfrac{\partial S_w}{\partial x}dx+\dfrac{\partial S_w}{\partial t}dt=0
\end{cases}
\eeq

Для того, чтобы система \eqref{Front} однородных и линейных (относительно частных производных водонасыщенности по координате и времени) уравнений имела отличное от нуля (нетривиальное) решение, необходимо, чтобы её определитель был равен нулю:
\beq
\left\lvert
\begin{matrix}
\varphi & \dfrac{\partial f_w}{\partial S_w}\dfrac{q_t}{A}\\\\
dt & dx
\end{matrix}
\right\rvert=0
\eeq

Откуда находим:
\beq\label{SaturationFrontVelocity}
\frac{dx}{dt}=\frac{q_t}{A\varphi}\frac{\partial f_w}{\partial S_w},
\eeq
где производные $\dfrac{dx}{dt}$ и $\dfrac{\partial f_w}{\partial S_w}$ вычисляются при постоянном значении водонасыщенности $S_{w0}$.
\\

Уравнение \eqref{SaturationFrontVelocity} можно интерпретировать следующим образом: фронт с постоянной заданной насыщенностью движется с постоянной скоростью, пропорциональной $q_t/(A\cdot\varphi)$.
\\

Итак, скорость движения фронта с данной водонасыщенностью $S_{w0}$:
\beq
v|_{S_{w0}}=\frac{dx}{dt}\bigg|_{S_{w0}}=\frac{q_t}{A\varphi}\frac{\partial f_w}{\partial S_w}\bigg|_{S_{w0}}
\eeq

Т.е. скорость движения фронта с данной водонасыщенностью зависит от характера производной функции Баклея-Леверетта по насыщенности и конкретного значения этой насыщенности на рассматриваемом фронте.

\textbf{Важно!} Разные насыщенности движутся с разной скоростью.

\end{document}
