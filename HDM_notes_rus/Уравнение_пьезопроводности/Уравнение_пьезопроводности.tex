\documentclass[a4paper,14pt]{extarticle}
\renewcommand{\baselinestretch}{1.2}
\usepackage[utf8]{inputenc}
\usepackage[T2A, T1]{fontenc}
\usepackage[english, russian]{babel}

\usepackage{fontspec}
\setmainfont{Times New Roman}
\usepackage{setspace, amsmath}
\usepackage{amssymb}
\usepackage{dsfont}
\usepackage{epsfig}
\usepackage{pdfpages}

\makeatletter
\let\@fnsymbol\@arabic
\makeatother

\usepackage{geometry}
\geometry{
a4paper,
total={180mm, 257mm},
left=15mm,
top=10mm,
}

\usepackage{systeme}
\usepackage{skak}
\usepackage{mathtools}
\usepackage{unicode-math}
\usepackage{array}
\usepackage{makecell}
\usepackage{subfiles}
\usepackage{hyperref}
\hypersetup{pdfstartview=FitH, linkcolor=black, urlcolor=blue, colorlinks=true}
\usepackage{framed}
\usepackage{graphicx}
\usepackage{caption}
\usepackage{subcaption}
\usepackage{color}
\usepackage{chngcntr}
\usepackage{tikz}
\usepackage{csquotes}
\usepackage{fancyhdr}
\usepackage{fancyvrb}
\usepackage{comment}
\usepackage{adjustbox}
\usepackage[breakable, skins]{tcolorbox}


\usepackage{float}
\floatstyle{plaintop}
\usepackage{enumitem}
\setlength{\parindent}{0pt}
%\setcounter{section}{-1}

\graphicspath{{./img/}}
\newcommand{\myPictWidth}{.95\textwidth}
\newcommand{\phm}{\phantom{-}}
\newcommand{\beq}{\begin{equation}}
\newcommand{\eeq}{\end{equation}}


\begin{document}

\section*{Уравнение пьезопроводности}

Уравнение пьезопроводности -- основное уравнение подземной гидродинамики, которое описывает процесс фильтрации жидкости в пористых средах.

Уравнение пьезопроводности строится на трёх уравнениях: на уравнении неразрывности потока (т.е. уравнении переноса массы, записанном в дифференциальной форме), на законе Дарси и на соотношении для сжимаемости флюида.
\\

Набор исходных уравнений для вывода уравнения пьезопроводности:
\begin{itemize}
	\item неразрывность потока
	\beq\label{Continuity}
	\frac{\partial\left(\rho_{\!f}\,\varphi\right)}{\partial t}+\pmb{\nabla}\cdot\left(\rho_{\!f}\,\varphi \, \pmb{v_f}\right)=q_f(\pmb{x}),
	\eeq
	где $\rho_{\!f}$ -- истинная плотность флюида; $\varphi$ -- пористость; $\pmb{v_{\!f}}$ -- истинная скорость флюида; $q_f(\pmb{x})$ -- источниковое слагаемое
	\item закон Дарси
	\beq\label{Darcy}
	\pmb{W}=-\frac{k}{\mu_f}\cdot\pmb{\nabla} p,
	\eeq
	где $\pmb{W}=\varphi\!\left(\pmb{v_{\!f}}-\pmb{v_{\!s}}\right)$ -- потоковая скорость флюида; $k$ -- проницаемость пласта; $\mu_{\!f}$ -- вязкость флюида; $\nabla p$ -- градиент давления
	\item соотношение на сжимаемость флюида
	\beq\label{Compressibility}
	p-p_0=K_{\!f}\frac{\rho_{\!f}-\rho_{\!f}^0}{\rho_{\!f}^0},
	\eeq
	где $K_{\!f}$ -- объёмный модуль упругости флюида (величина, обратная коэффициенту сжимаемости)
\end{itemize}
\ \\
Этих трёх уравнений достаточно, чтобы вывести уравнение пьезопроводности в случае недеформируемого пласта.

Если пласт упругий, то дополнительно необходимо использовать 3 уравнения пороупругости, накладывающие условия на тензор полных напряжений, тензор полных деформаций и пористость пласта.

Далее будет представлено 2 вывода уравнения пьезопроводности в случае недеформируемого пласта и 1 вывод уравнения пьезопроводности в случае упругого пласта.
\newpage

\textbf{Вывод уравнения пьезопроводности в векторной форме}

В предположении неподвижности скелета ($\pmb{v_{\!s}}\approx \pmb{0}$ и $\varphi(t)=\textrm{const}$) верно равенство $\pmb{W}\approx\varphi\, \pmb{v_{\!f}}$.
Подставляя это выражение в закон Дарси \eqref{Darcy}, получаем:
\beq\label{DarcyWithSkeletNotMoving}
\varphi\,\pmb{v_{\!f}}=-\frac{k}{\mu_{\!f}}\cdot\pmb{\nabla} p
\eeq

Условие сжимаемости флюида \eqref{Compressibility} перепишем в дифференциальной форме:
\beq\label{CompressibilityDiff}
\frac{\partial p}{\partial t}=\frac{K_{\!f}}{\rho_{\!f}^0}\frac{\partial\rho_{\!f}}{\partial t}
\eeq

Учитывая предположение о неподвижности скелета, перепишем уравнение неразрывности потока \eqref{Continuity}:
\beq\label{ContinuityWithSkeletNotMoving}
\varphi\frac{\partial\rho_{\!f}}{\partial t}+\pmb{\nabla}\cdot\left(\rho_{\!f}\,\varphi \, \pmb{v_f}\right)=q_f(\pmb{x})
\eeq

Подставляя \eqref{DarcyWithSkeletNotMoving} и \eqref{CompressibilityDiff} в \eqref{ContinuityWithSkeletNotMoving}, при отсутствии источникового слагаемого ($q_f(\pmb{x})=0$) получаем:
\beq
\varphi\frac{\rho_{\!f}^0}{K_{\!f}}\frac{\partial p}{\partial t}-\pmb{\nabla}\cdot\left(\rho_{\!f}\frac{k}{\mu_{\!f}}\pmb{\nabla} p\right)=0
\eeq

При дополнительном условии слабосжимаемости флюида ($\rho_f\approx\rho_f^0=\textrm{const}$) получаем:
\beq
\frac{\partial p}{\partial t}=\frac{kK_{\!f}}{\mu_{\!f}\,\varphi}\,\pmb{\nabla}^2p
\eeq

Это уравнение пьезопроводности (без упругости пласта), полученное в приближении слабосжимаемого флюида, неподвижного и недеформируемого пласта.

Введя коэффициент пьезопроводности
\beq
\kappa=\dfrac{kK_f}{\mu_{\!f}\,\varphi}=\dfrac{k}{\mu_{\!f}c_{\!f}},
\eeq
где $c_{\!f}=\varphi/K_{\!f}$ -- сжимаемость флюида, можем переписать уравнение пьезопроводности в следующем виде:
\beq
\frac{\partial p}{\partial t}=\kappa\,\pmb{\nabla}^2p
\eeq

Коэффициент пьезопроводности пласта $\kappa$ -- коэффициент, характеризующий темпы распределения пластового давления в условиях упругого режима, равный отношению проницаемости пласта к произведению вязкости жидкости и её сжимаемости.
\newpage

\textbf{Вывод уравнения пьезопроводности в покомпонентной форме с обезразмериванием (недеформируемый пласт, но учитываем зависимость пористости от давления)}

Запишем \textbf{ЗСМ для флюида}:
\beq
\frac{\partial r_{\!f}}{\partial t}+\partial_i\!\left(r_{\!f}\, v_i^f\right)=0,
\eeq
где $r_{\!f}=\varphi\rho_{\!f}$ -- эффективная плотность флюида; $v_i^f$ -- истинная скорость флюида.

\textbf{Закон Дарси в дифференциальной форме}:
\beq\label{DarcyDiffAdvanced}
W_i=-\frac{k_{ij}}{\mu}\,\partial_{\!j}p,
\eeq
где $W_i=\varphi v_i^f$ -- потоковая относительная скорость флюида (при неподвижном скелете), $k_{ij}$ -- тензор абсолютной проницаемости пласта, $\mu$ -- вязкость флюида.

Учитывая связь эффективной и истинной плотностей ($r_{\!f}=\varphi\rho_{\!f}$), перепишем ЗСМ для флюида:
\beq\label{ContinuityAdvanced}
\frac{\partial\!\left(\rho_{\!f}\,\varphi\right)}{\partial t}+\partial_i\!\left(\rho_{\!f}\,\varphi v_i^f\right)=0
\eeq

Подставляя \eqref{DarcyDiffAdvanced} в \eqref{ContinuityAdvanced}, получаем:
\beq\label{GeneralPiezo}
\frac{\partial\!\left(\rho_{\!f}\,\varphi\right)}{\partial t}-\partial_i\!\left(\rho_{\!f}\,\frac{k_{ij}}{\mu}\,\partial_{\!j} p\right)=0
\eeq


Замыкающее соотношение (\textbf{связь плотности флюида и давления}):
\beq\label{Zam1}
\rho_{\!f}=\rho_{\!f}^0\left(1+c_{\!f}\left(p-p_0\right)\right),
\eeq
где $c_{\!f}$ -- сжимаемость флюида (1/Па).


Замыкающее соотношение (\textbf{связь пористости и давления}):
\beq\label{Zam2}
\varphi=\varphi^0+c_{\text{п}}\left(p-p_0\right),
\eeq
где $c_{\text{п}}$ -- сжимаемость пор (не равно сжимаемости породы).


Производные по времени и пространству замыкающего соотношения \eqref{Zam1} запишутся в следующем виде:
\beq\label{DiffZam1}
\frac{\partial\rho_{\!f}}{\partial t}=c_f\rho_{\!f}^0\frac{\partial p}{\partial t}
\eeq

\beq\label{GradZam1}
\partial_i\rho_{\!f}=c_f\rho_{\!f}^0\partial_i p
\eeq

А производные по времени и пространству замыкающего соотношения \eqref{Zam2} примут вид:
\beq\label{DiffZam2}
\frac{\partial\varphi}{\partial t}=c_\text{п}\frac{\partial p}{\partial t}
\eeq

\beq\label{GradZam2}
\partial_i\varphi=c_\text{п}\partial_i p
\eeq


Раскрывая производные произведений в \eqref{GeneralPiezo}, получаем:
\beq\label{OpenGeneralContinuity}
\frac{\partial\rho_{\!f}}{\partial t}\varphi+\rho_{\!f}\frac{\partial\varphi}{\partial t}-\frac{k_{ij}}{\mu}\,\partial_{\!j} p\,\partial_i\rho_f-\rho_{\!f}\,\partial_{\!j} p\,\,\partial_i\!\!\left(\frac{k_{ij}}{\mu}\right)-\rho_{\!f}\frac{k_{ij}}{\mu}\left(\partial_i\partial_{\!j} p\right)=0
\eeq

Подставляя \eqref{DiffZam1}, \eqref{GradZam1}, \eqref{DiffZam2} и \eqref{GradZam2} в \eqref{OpenGeneralContinuity}, получаем:
\begin{multline}\label{Expanded}
c_{\!f}\,\rho_{\!f}^0\frac{\partial p}{\partial t}\varphi+\rho_{\!f} c_\text{п}\frac{\partial p}{\partial t}-\frac{k_{ij}}{\mu}\,\partial_{\!j} p\,c_{\!f}\,\rho_{\!f}^0\,\partial_i p-\frac{\rho_{\!f}}{\mu}\partial_{\!j} p\,\partial_i k_{ij}+\\+\rho_{\!f}\,\partial_{\!j} p\,k_{ij}\,\frac{\partial\mu}{\partial p}\frac{1}{\mu^2}\partial_i p-\rho_{\!f}\frac{k_{ij}}{\mu}\!\left(\partial_i\partial_{\!j} p\right)=0
\end{multline}


Перед анализом физических уравнений всегда делают масштабный анализ, чтобы понять, какие слагаемые в уравнении важны, а какие не важны (пример: уравнение Навье-Стокса с числами Струхаля, Эйлера, Рейнольдса, Фруда).

\textbf{Замечание.} ГДМ симуляторы не решают уравнение пьезопроводности в классическом виде, а решают закон сохранения массы, в который они подставляют закон Дарси.

Далее необходимо выделить характерные масштабные факторы, обезразмерив каждую из функций в уравнении.

Введём безразмерные давление $\tilde{p}$, расстояние $\tilde{r}$, проницаемость $\tilde{k}_{ij}$ и вязкость $\tilde{\mu}$ такие, что:
\beq
p=\tilde{p}\cdot p_0,\,\,\,\vec{r}=\tilde{r}\cdot L,\,\,\,k_{ij}=\tilde{k}_{ij}\cdot k_0,\,\,\,\mu=\tilde{\mu}\cdot\mu_0,
\eeq
где $p_0$ -- пластовое давление, $L$ -- некое характерное расстояние (например, расстояние между скважинами), $k_0$ -- некая характерная проницаемость, $\mu_0$ -- некая характерная вязкость.

Все безразмерные функции (с волной) порядка единицы.

Перепишем \eqref{Expanded} в введённых безразмерных величинах, разделив обе части этого уравнения на $\rho_f^0$:
\begin{multline}
\frac{\partial p}{\partial t}\left(\varphi c_{\!f}+\frac{\rho_{\!f}}{\rho_{\!f}^0}\cdot c_\text{п}\right)-\frac{k_0}{\mu_0}\frac{p_0^2}{L^2}c_{\!f}\frac{\tilde{k}_{ij}}{\tilde{\mu}}\,\tilde{\partial}_i\tilde{p}\,\tilde{\partial}_{\!j}\tilde{p}-\frac{\rho_{\!f}}{\rho_{\!f}^0}\frac{k_0\,p_0}{\mu_0L^2}\frac{\tilde{k}_{ij}}{\tilde{\mu}}\,\tilde{\partial}_{\!j}\tilde{p}\,\tilde{\partial}_i\tilde{k}_{ij}+\\+\frac{\rho_{\!f}}{\rho_{\!f}^0}\frac{p_0\,k_0}{L^2\mu_0}\,\tilde{\partial}_{\!j}\tilde{p}\,\tilde{k}_{ij}\frac{\partial\tilde{\mu}}{\partial\tilde{p}}\frac{1}{\tilde{\mu}^2}\,\tilde{\partial}_i\tilde{p}-\frac{\rho_{\!f}}{\rho_{\!f}^0}\frac{k_0}{\mu_0}\frac{p_0}{L^2}\,\frac{\tilde{k}_{ij}}{\tilde{\mu}}\left(\tilde{\partial}_i\tilde{\partial}_{\!j}\tilde{p}\right)=0
\end{multline}

Вынесли все масштабные множители. Далее делим обе части уравнения на множитель перед старшей производной $\left(\text{на }\frac{k_0\,p_0}{\mu_0\,L^2}\right)$, т.е. обезразмериваем уравнение:
\begin{multline}\label{PiezoEqDiv}
\frac{\mu_0L^2}{k_0p_0}\cdot\frac{\partial p}{\partial t}\left(\varphi c_{\!f}+\frac{\rho_{\!f}}{\rho_{\!f}^0}\cdot c_\text{п}\right)-p_0c_{\!f}\frac{\tilde{k}_{ij}}{\tilde{\mu}}\,\tilde{\partial}_i\tilde{p}\,\tilde{\partial}_{\!j}\tilde{p}-\frac{\rho_{\!f}}{\rho_{\!f}^0}\frac{\tilde{k}_{ij}}{\tilde{\mu}}\,\tilde{\partial}_{\!j}\tilde{p}\,\tilde{\partial}_i\tilde{k}_{ij}+\\+\frac{\rho_{\!f}}{\rho_{\!f}^0}\frac{\partial\tilde{\mu}}{\partial\tilde{p}}\frac{1}{\tilde{\mu}^2}\tilde{k}_{ij}\,\tilde{\partial}_{\!j}\tilde{p}\,\tilde{\partial}_i\tilde{p}-\frac{\rho_{\!f}}{\rho_{\!f}^0}\frac{\tilde{k}_{ij}}{\tilde{\mu}}\left(\tilde{\partial}_i\tilde{\partial}_{\!j}\tilde{p}\right)=0
\end{multline}


Сделаем 3 важных приближения:
\begin{enumerate}
	\item $p_0 c_f\ll 1$ (прикинем: сжимаемость воды порядка $10^{-5}\text{ атм}^{-1}=10^{-10}\text{ Па}^{-1}$; характерные значения давлений на глубинах, равных нескольким километрам, составляют сотни атмосфер; таким образом, произведение порядка $10^{-3}$, что много меньше единицы; но такое приближение не работает для газа: для него рассматриваемое произведение порядка единицы); это приближение фактически равносильно приближению $\rho_f\approx\rho_f^0$;
	\item $\tilde{\partial}_i\tilde{k}_{ij}\ll 1$ (считаем, что на характерном масштабе задачи по данному направлению проницаемость изменяется незначительно, не больше 10 процентов);
	\item $\dfrac{\partial\tilde{\mu}}{\partial\tilde{p}}\ll 1$ (считаем, что отмасштабированный график проницаемости от давления пологий -- этот факт подтверждается экспериментально -- вязкость слабо зависит от давления)
\end{enumerate}

Тогда уравнение \eqref{PiezoEqDiv} перепишется в следующем виде (убрали слагаемые с пренебрежимо малыми множителями в рамках сделанных приближений):
\beq
\frac{\mu_0L^2}{k_0p_0}\cdot\frac{\partial p}{\partial t}\underbrace{\left(\varphi c_{\!f}+c_\text{п}\right)}_{c_t}-\frac{\tilde{k}_{ij}}{\tilde{\mu}}\,\tilde{\partial}_i\tilde{\partial}_{\!j} \tilde{p}=0
\eeq

Равносильно:
\beq
\frac{\partial p}{\partial t}\underbrace{\left(\varphi c_{\!f}+c_\text{п}\right)}_{c_t}-\frac{k_0p_0}{\mu_0L^2}\cdot\frac{\tilde{k}_{ij}}{\tilde{\mu}}\,\tilde{\partial}_i\tilde{\partial}_{\!j} \tilde{p}=0
\eeq

Вернёмся от безразмерных функций с волной к обычным функциям:
\beq
\frac{\partial p}{\partial t}\underbrace{\left(\varphi c_{\!f}+c_\text{п}\right)}_{c_t}-\frac{k_{ij}}{\mu}\,\partial_i\partial_{\!j} p=0
\eeq
(заметим, что если есть анизотропия проницаемости, то лапласиана в уравнении не будет; вместо него будет честный $\partial_i\partial_{\!j}p$).


Получаем классическое уравнение пьезопроводности:
\beq
\frac{\partial p}{\partial t}-\frac{k_{ij}}{\mu c_t}\partial_i\partial_{\!j} p=0,
\eeq
где $c_t$ -- это полная сжимаемость.

\textbf{Замечание.} Но есть литература, в которой $c_t=c_{\!f}+\dfrac{c_\text{п}}{\varphi}=c_{\!f}+c_{\text{породы}}$, тогда уравнение пьезопроводности будет выглядеть так:
\beq
\frac{\partial p}{\partial t}-\frac{k_{ij}}{\mu\varphi c_t}\partial_i\partial_{\!j} p=0
\eeq


\textbf{Пусть тензор проницаемости изотропен} $k_{ij}=k_0\cdot\delta_{ij}$, тогда:
\beq
\frac{\partial p}{\partial t}-\frac{k_0}{\mu c_t}\delta_{ij}\,\partial_i\partial_{\!j} p=0\Leftrightarrow\frac{\partial p}{\partial t}-\frac{k_0}{\mu c_t}\Delta p=0
\eeq
(получили всем известный вид уравнения пьезопроводности).
\newpage

\textbf{Задача по моделированию фильтрации жидкости в пласте с помощью уравнения пьезопроводности.}

Полубесконечный горизонтальный пористый пласт имеет вид прямоугольного параллелепипеда, все грани которого (кроме левой) непроницаемы для жидкости.
Пласт насыщен жидкостью с коэффициентом объёмной сжимаемости $\beta=1\cdot 10^{-10}\text{ Па}^{-1}$, давление которой во всех точках в начальный момент времени постоянно и равно $p_0=5\text{ МПа}$.
На левой границе пласта давление в момент времени $t=0$ мгновенно уменьшается до $p_e=1\text{ МПа}$ и в дальнейшем поддерживается постоянным.

Определить значение давления в момент времени $t=100\text{ с}$ в точке пласта, отстоящей на расстоянии $x=10\text{ м}$ от его левой границы, если зависимость скорости течения жидкости в пласте от давления описывается линейным законом Дарси.
Известно, что пористость пласта $\varphi=0.1$, проницаемость пласта $k=10^{-14}\text{ м}^2$ и динамическая вязкость жидкости $\mu=10^{-3}\text{ Па}\cdot\text{с}$
\\

\textbf{Решение задачи.}

Решение задачи сводится к решению начально-краевой задачи для уравнения пьезопроводности:
\beq\label{InitialTask}
\begin{cases}
\dfrac{\partial p}{\partial t}=\overbrace{\left(\dfrac{k}{\varphi\mu\beta}\right)}^{\substack{\text{коэффициент}\\\text{пьезопро-ти }\kappa}}\dfrac{\partial^2 p}{\partial x^2}\\
p(x,0)=p_0 \,(\text{при }x\geqslant0)\\
p(0,t)=p_e \,(\text{при }t>0)\\
p(\infty, t)=p_0 \,(\text{при }t>0)
\end{cases}
\eeq

Введём новую безразмерную переменную
\beq\label{NewVar}
\xi=\dfrac{x}{2\sqrt{\kappa t}}
\eeq
и сведём задачу к нахождению давления $p$, зависящего только от новой переменной.
\\

По правилу дифференцирования сложных функций получим:
\beq
\frac{\partial p}{\partial t}=\frac{d p}{d\xi}\frac{\partial\xi}{\partial t}=\frac{d p}{d\xi}\left(-\frac{\xi}{2t}\right)
\eeq
\beq
\frac{\partial p}{\partial x}=\frac{d p}{d\xi}\frac{\partial\xi}{\partial x}=\frac{d p}{d\xi}\frac{1}{2\sqrt{\kappa t}}
\eeq
\beq
\frac{\partial^2p}{\partial x^2}=\frac{\partial}{\partial x}\left(\frac{\partial p}{\partial x}\right)=\frac{\partial}{\partial x}\left(\frac{1}{2\sqrt{\kappa t}}\frac{dp}{d\xi}\right)=\frac{1}{2\sqrt{\kappa t}}\frac{d^2p}{d\xi^2}\frac{\partial\xi}{\partial x}=\frac{1}{4\kappa t}\frac{d^2p}{d\xi^2}
\eeq
\\

Подставляя найденные значения производных в уравнение пьезопроводности, получим обыкновенное дифференциальное уравнение:
\beq\label{ODE}
-2\cdot\xi\cdot\frac{dp}{d\xi}=\frac{d^2p}{d\xi^2}
\eeq

Начальные и граничные условия исходной начально-краевой задачи \eqref{InitialTask} перепишутся в следующем виде:
\beq\label{ODEBoundaries}
\begin{cases}
p|_{\xi=0}=p_e\\
p|_{\xi\rightarrow\infty}=p_0
\end{cases}
\eeq
\ \\

Для решения полученного обыкновенного дифференциального уравнения второго порядка \eqref{ODE} введём новую переменную:
\beq
p_\xi=\frac{dp}{d\xi}
\eeq
и перепишем уравнение \eqref{ODE} в следующем виде:
\beq
-2\cdot\xi\cdot p_\xi=\frac{dp_\xi}{d\xi}
\eeq

Разделив переменные и проинтегрировав, получим:
\beq
\ln{p_\xi}=-\xi^2+\ln{C_1}
\eeq
или
\beq
\frac{dp}{d\xi}=C_1\exp\left(-\xi^2\right)
\eeq

Интегрируя данное уравнение, получим общее решение дифференциального уравнения \eqref{ODE}:
\beq
p=C_1\cdot\int\limits_{0}^{\xi}\exp{\left(-\xi^2\right)d\xi}+C_2
\eeq

Константы интегрирования $C_1$ и  $C_2$ определим из условий \eqref{ODEBoundaries}:
\beq
\begin{cases}
C_2=p_e\\\\
C_1=\dfrac{p_0-p_e}{\int\limits_{0}^{\infty}{\exp{\left(-\xi^2\right)d\xi}}}
\end{cases}
\eeq

Получаем следующее решение:
\beq
p=p_e+\dfrac{p_0-p_e}{\int\limits_{0}^{\infty}{\exp{\left(-\xi^2\right)d\xi}}}\cdot\int\limits_{0}^{\xi}\exp{\left(-\xi^2\right)d\xi}
\eeq

Известно, что $\displaystyle{}\int\limits_{0}^{\infty}{\exp{\left(-\xi^2\right)d\xi}}=\frac{\sqrt{\pi}}{2}$, поэтому
\beq
p=p_e+\frac{2\left(p_0-p_e\right)}{\sqrt{\pi}}\cdot\int\limits_{0}^{\xi}\exp{\left(-\xi^2\right)d\xi}
\eeq

Знаем, что $\displaystyle{}\dfrac{2}{\sqrt{\pi}}\int\limits_{0}^{\xi}\exp{\left(-\xi^2\right)}d\xi=\text{erf}\left(\xi\right)$ -- это функция ошибок Гаусса, поэтому:
\beq
p=p_e+\left(p_0-p_e\right)\text{erf}\left(\xi\right)
\eeq
или, подставляя выражение \eqref{NewVar} для $\xi$, получаем:
\beq
p=p_e-\left(p_e-p_0\right)\,\text{erf}\!\left(\dfrac{x}{2\sqrt{\dfrac{k}{\varphi\mu\beta}\,t}}\right)
\eeq

Подставляя в это выражение исходные данные задачи, получим, что давление в указанной точке и в указанный момент времени примерно равно $3.082\text{ МПа}$.

\textbf{Замечание.} Но обычно уравнение пьезопроводности решают не аналитически, а численно, так как у реальных залежей геометрия гораздо более сложная.

\newpage

\textbf{Вывод уравнения пьезопроводности для "<упругого пласта">}

Для \textbf{упругого изотропного пласта} можем записать известные соотношения пороупругости:
\begin{itemize}[parsep=-1pt]
\item на тензор полных напряжений \\
\beq
\pmb{T}=\sigma^0\pmb{I}+\left(\lambda I_1(\pmb{\varepsilon})-b\Delta p\right)\pmb{I}+2\mu\pmb{\varepsilon},
\eeq
где $\pmb{T}$ -- тензор полных напряжений; $\pmb{I}$ -- единичный тензор; $\pmb{\varepsilon}$ -- тензор полных деформаций; $\lambda=K-2G/3$ и $\mu=G$ -- константы (параметры) Ляме; $K$ -- модуль всестороннего сжатия; $G$ -- модуль сдвига; $I_1(\pmb{\varepsilon})$ -- след тензора полных деформаций; $b$ -- константа Био; $\Delta p$ -- изменение давления; $\sigma^0$ -- начальное напряжение
\item на пористость \\
\beq
\varphi = \varphi_0+bI_1(\pmb{\varepsilon})+\dfrac{1}{N}\Delta p,
\eeq
где $\varphi_0$ -- начальная пористость; $b$ -- константа Био; $I_1(\pmb{\varepsilon})$ -- след тензора полных деформаций; $N$ -- модуль Био; $\Delta p$ -- изменение давления.
\item условие равновесия \\
\beq
\pmb{\nabla}\cdot\pmb{T}=\pmb{0}
\eeq
\end{itemize}

Для \textbf{флюида} запишем:
\begin{itemize}[parsep=-1pt]
	\item закон Дарси \\
	\beq
	\pmb{W}=-\frac{k}{\mu_f}\cdot\pmb{\nabla} p,
	\eeq
	где $\pmb{W}=\varphi\left(\pmb{v_{\!f}}-\pmb{v_{\!s}}\right)$; $k$ -- проницаемость пласта; $\mu_{\!f}$ -- вязкость флюида; $\pmb{\nabla} p$ -- градиент давления
	\item соотношение на сжимаемость флюида
	\beq
	p-p_0=K_{\!f}\frac{\rho_{\!f}-\rho_{\!f}^0}{\rho_{\!f}^0},
	\eeq
	где $\dfrac{1}{K_{\!f}}$ -- сжимаемость флюида ($K_{\!f}$ -- объёмный модуль упругости флюида)
	\item уравнение неразрывности потока при отсутствии источникового слагаемого (уравнение переноса массы, записанное в дифференциальной форме):
	\beq
	\frac{\partial\left(\rho_{\!f}\,\varphi\right)}{\partial t}+\pmb{\nabla}\cdot\left(\rho_{\!f}\,\varphi\,\pmb{v_{\!f}}\right)=0
	\eeq
\end{itemize}

Из \textbf{уравнения неразрывности} получаем:
\beq
\varphi_0\frac{\partial\rho_{\!f}}{\partial t}+\rho_0\frac{\partial\varphi}{\partial t}+\rho_0\pmb{\nabla}\cdot\pmb{W}+\rho_0\varphi_0\frac{\partial I_1(\pmb{\varepsilon})}{\partial t}=0,
\eeq
где
\beq
\frac{\partial I_1(\pmb{\varepsilon})}{\partial t}\equiv\pmb{\nabla}\cdot\pmb{v_s}
\eeq

А дальше через ряд свёрток и всяких операций получаем:
\beq
b\dot{I}_1(\pmb{\varepsilon})+\left(\frac{1}{N}+\frac{\varphi_0}{K_{\!f}}\right)\dot{p}=\frac{k}{\mu_{\!f}}\pmb{\nabla}^2p
\eeq

В осесимметричном случае при условии отсутствия деформации на бесконечности получаем:
\beq
\dot{p}=a\pmb{\nabla}^2p,
\eeq
где
\beq
a=\frac{kM}{\mu_{\!f}}\text{ и } M=\frac{b\left(b+\varphi\right)}{\lambda+2\mu}+\frac{1}{N}+\frac{\varphi}{K_{\!f}}
\eeq

\end{document}
